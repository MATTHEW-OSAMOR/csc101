	\documentclass{article}
	
	\title{My Historical Perspective Of 5 Programming Languages} 
	\date{10-41-2021}
	\author{Matthew Osamor}
	\begin{document}
		\maketitle
			\section{JAVA}
		First of all I would like to give a little history on java: it first came up on the 13th of may 1995.
		It was actually designed for interactive television, but it turned out to be much more than that.it had the capacity to do much more than what the cable tv industry could do as of then, so it was put into other uses which will be explained below.
		\subsection{CREATOR:}
		Creator: james gosling and his team which are also known as the (the green team).
		\subsubsection{USES OF JAVA}
		Now java is used for all sorts of Various of things such as games, computer programming, e-businesses.
		\subsubsection{SUITABLE I.D.E.S FOR JAVA}
		IDES such as IntelliJ IDEA and net beans can be used for java programming 
		\subsubsection{SIMILAR PROGRAMS LIKE JAVA}
		Java is a lot like lisp or small talk.
			\section{BASIC}
		Basic which stands for beginners all-purpose instruction code started or rather appeared on the 1st of may 1964.
		It was created to enable all types of people/students (not just the scientists  or mathematicians)use computers effectively.
		\subsection{CREATOR}
		created by:john g kemney and Thomas e kurtz 
		\subsubsection{USES OF BASIC}
		Basic can be used to interact with a program's source code, or source
		\subsubsection{I.D.E.S FOR USED FOR BASIC}
		IDES such as codelite and blue fish can be used for BASIC PROGRAMMING
		\subsubsection{OTHER PROGRAMMING LANGUAGES LIKE BASIC}
		Basic has other related programs like tiny basic and Microsoft basic
		\section{PYTHON}
		Python is a language which started around December 1989. 
		Its purpose for creation was to be able to cope with and handle the amoeba operating system (the amoeba operating system is a project which creates a time sharing system that makes an entire network of computers seem like a single machine to a user)
		\subsection{CREATOR}
		Creator: guido van Rossum is python’s creator
		\subsubsection{USES OF PYTHON}
		Python can be used for various things like building websites,
		software and for conduction of most analysis
		\subsubsection{I.D.E.S FOR PYTHON}
		i.d.e.s for python are spyder, pycharm, eclipse, etc.
		\subsubsection{OTHER PROGRAMS LIKE PYTHON}
		Other programs like java, php, ruby are related to python
			\section{C++}
		C++ is an object oriented programming language (this means it is a language which deals with “objects” that have or contain data/code which are in form of fields). It was created as an extension to the c programming language. It was created in 1985.
		\subsection{CREATOR}
		Creator :Bjarne Stroustrup
		\subsubsection{USES OF C++}
		Honestly almost any coding or software has c++ imbedded in it but ill just name a few such as google, adobe, hp(Hewlett Packard),ibm, etc.
		\subsubsection{I.D.E.S FOR C++}
		Like I said earlier, c++ can go with almost anything so i.d.e.s like eclipse, netbeans, etc can be used as suitable ides
		\subsubsection{OTHER PROGRAMS LIKE C++}
		Programs like c, python, java and ruby are related to c++
			\section{RUBY}
		Ruby is a programing language which started between the years of 1990 and 1999. 
		\subsection{USES OR RUBY}
		It was made as a programming language which made coding and programing fun which would lead for more productivity. (Honestly I couldn’t find much on ruby without it being plagiarized)
		\subsubsection{CREATOR}
		Creator: yukihiro matsumoto
		Java  can be used for things like building web applications, data analysis, digital frame work etc.
		\subsubsection{I.D.E.S FOR RUBY}
		Ides like ruby mine, aptana studio, emacs and vim function with ruby
		\subsubsection{OTHER PROGRAMS FOR RUBY}
		Programs related to ruby are perl and python
		\end{document}